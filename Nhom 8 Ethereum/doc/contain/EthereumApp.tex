\documentclass[../cnd.tex]{subfiles}

\subsubsection{Hiện tại}
\paragraph{Hệ thống thanh toán}\mbox{}\\

Nếu xét về khía cạnh ứng dụng hệ thống thanh toán thì Ethereum cũng tương tự như Bitcoin. Nếu đồng tiền Bitcoin là một ứng dụng của chuỗi khối Bitcoin thì Ethereum cũng là một ứng dụng của chuỗi khối Ethereum. Ethereum từ khi ra đời đã có nhiều cuộc tranh cãi cho rằng nó có thể sử dụng để lưu trữ giá trị đồng tiền. Hầu hết các thanh toán trong mạng lưới Ethereum được xác nhận bởi các thợ đào mỏ và được ghi chép, lữu trữ vào cuốn sổ cái công khai, như nhau cả.

\paragraph{Đầu tư vàng}\mbox{}\\

Mạng lưới Ethereum đã được các kỹ sư của Digix sử dụng để xây dựng ứng dụng đầu tư vàng. Digix có thể được sử dụng để mua những token vàng bằng tiền mặt hoặc sử dụng đồng tiền ether. Những token này được mã hóa rất phức tạp và liên kết chặt chẽ với các thợ mỏ vàng tại Singapore. Trường hợp Digix bị phá sản, bạn vẫn thể đổi các token vàng này thành vàng thật mà không cần tới bên trung gian nào như ngân hàng, các nhà môi giới hay tiệm vàng.
Bạn có thể truy cập tại: \url{https://digix.global}.

\paragraph{Gây quỹ công cộng}\mbox{}\\

Những tổ chức như Indiegogo (\url{www.indiegogo.com}) hay Kickstarter (\url{www.kickstarter.com}) đã sử dụng mạng lưới của Ethereum nhằm xây dựng và phát triển hệ thống gây quỹ cộng đồng. Việc cần làm chỉ là xác định ý tưởng và mục tiêu cho nguồn quỹ, khi thành công Kickstarter sẽ lấy 5\% từ nguồn quỹ, số còn lại sẽ chuyển đến các quỹ starup. Mạng dưới chuỗi khối Ethereum được các nhóm khởi nghiệp sử dụng để đặt mục tiêu cho ngân sách, việc còn lại sẽ được thực hiện bởi các hợp đồng thông minh nếu thành công.

\paragraph{Quản lý tài chính doanh nghiệp}\mbox{}\\

Tháng 5/2016, Quỹ cộng đồng lớn nhất trên thế giới có tên gọi The DAO được thành lập. Bản chất của The DAO là một quỹ đầu tư mạo hiểm dựa trên hệ thống biểu quyết phân cấp nhờ hợp đồng thông minh từ đó đưa ra những quyết định đầu tư hợp lý. Đây được coi là một thí nghiệm mang tính cách mạng cho loài người. Tuy nhiên, dự án này vẫn chưa thể triển khai, nếu The DAO thành công, thế giới sẽ chứng kiến một viễn cảnh khác khi các doanh nghiệp sử dụng mô hình quản lý bằng chuỗi khối, việc này đồng nghĩa sẽ không có chức vụ Chủ tịch Hội đồng quản trị, Tổng giám đốc hay các Phó - Trưởng phòng nữa.

\subsubsection{Trong tương lai}
\paragraph{Internet của vạn vật}\mbox{}\\

Đây có thể nói là một lý tưởng mà con người luôn muốn hướng tới. Một khi đạt được nó, mọi vật trên thế giới này sẽ được kết nối lại với nhau mà không có sự tương tác giữa con người với máy tính hay con người với con người. Bằng cách sử dụng một thiết bị gọi là Ethereum Computer (\url{https://slock.it/ethereum_computer.html}), mọi vật hay tài sản đều được hệ thống hóa quản lý trong một không gian kỹ thuật số. 

Một ví dụ cụ thể:
Giả sử với một chiếc máy rút tiền ATM khi hết tiền thì sẽ cần sự can thiệp của con người để kiểm tra xem tình trạng máy và quyết định đưa thêm tiền vào. Chưa hết muốn thêm tiền thì con người phải ký hàng loạt các giấy tờ và làm báo cáo có liên quan. Thực sự mất thời gian và rất rườm rà, còn với Ethereum Computer, các máy ATM sẽ tự liên kết với hệ thống kế toán và thêm tiền vào máy mà không cần sự can thiệp của con người, đảm bảo được tính bảo mật, độ tin cậy cao nhất.

\paragraph{Thị trường dự đoán}\mbox{}\\

Gnosis (\url{https://gnosis.pm}) và Augur (\url{https://augur.net}) chính là 2 ứng dụng thị trường dự đoán dựa trên mạng lưới của Ethereum.

\paragraph{Lưu trữ web}\mbox{}\\

Swarm là một trong những dự án của đồng tiền ảo Ethereum được phát triển bởi Viktor Tron. Qua đó Swarm sẽ cung cấp dịch vụ lưu trữ web cho nhiều người. Một trang web phân cấp sẽ được lưu trữ ở mọi nơi trong cùng một thời điểm. Việc này giúp phát hiện mọi hành vi tấn công DdoS hay ai đó đang tìm cách phá hoại website và ngăn chặn nó không được thực hiện, trừ khi hành vi đó được cấp phép trên toàn bộ hệ thống chuỗi khối. Bên cạnh đó cũng hỗ trợ hay thậm chí ngăn chặn việc kiểm duyệt và đánh sập website bởi các thế lực Chính phủ.

\paragraph{Mạng xã hội}\mbox{}\\

Chắc hẳn bạn không thể không biết tới Mark Zuckerberg người đứng đầu của mạng xã hội Facebook lớn nhất thế giới. Có thể bạn chưa biết Mark Zuckerberg vẫn đọc tin nhắn của bạn hằng ngày, bạn có quan tâm về việc những tấm ảnh mà bạn đã xóa vẫn còn tồn tại trên hệ thống facebook? Facebook hay bất kỳ mạng xã hội nào khác cũng vậy, chúng đều được quản lý bởi một bộ máy trung tâm. Vì thế, mạng xã hội Akasha (\url{http://akasha.world}) được ra đời, một ứng dụng của mạng lưới Ethereum. Một khi Akasha hoạt động, bạn sẽ giải đáp được tất cả các câu hỏi liên quan đến tính riêng tư khi bạn tham gia vào mạng xã hội của mình.

\paragraph{Chuyển nguồn năng lượng}\mbox{}\\

Với ứng dụng Ethereum bạn có thể chuyển nguồn năng lượng từ tấm pin mặt trời sang cho nhà bên cạnh một cách hoàn toàn tự động khi đã có đủ năng lượng sử dụng hoặc bạn có thể bán nó nếu không muốn chia sẻ miễn phí. Thật tuyệt vời phải không nào? Ứng dụng TransActive Grid (\url{http://transactivegrid.net}) này sẽ giúp bạn sử dụng pin mặt trời một cách hiệu quả hơn.

\paragraph{Giấy kết hôn – di chúc}\mbox{}\\

Nghe có vẻ xa vời nhưng điều này hoàn toàn có thể được thực hiện với ứng dụng của Ethereum, bạn có thể lưu trữ giấy kết hôn hay một bản di chúc trên hệ thống chuỗi khối. Hợp đồng thông minh sẽ giúp bạn quản lý mọi điều khoản pháp lý có liên quan. Điều này sẽ sớm xuất hiện trong tương lai gần.

\paragraph{Thị trường tài chính – Bầu cử – Bất động sản}\mbox{}\\

Để mọi việc trở nên công bằng hơn, không có thao túng, gian lận, đã đến lúc kết với nó với chuỗi khối. Mạng lưới này sẽ xác định mọi thứ liên quan, tất cả những gì phải làm và quản lý chúng một cách minh bạch nhất.