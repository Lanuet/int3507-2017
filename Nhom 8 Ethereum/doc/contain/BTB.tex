\documentclass[../cnd.tex]{subfiles}

Các chuỗi khối kinh điển là một con đường từ gốc đến lá thông qua toàn bộ khối cây. Để có sự nhất trí về con đường đó, khái niệm chúng ta xác định con đường có tính toán nhiều nhất được thực hiện theo nó, hoặc, con đường nặng nhất. Rõ ràng một yếu tố giúp xác định được con đường nặng nhất là số khối của lá, tương đương với số khối, không tính đến khối gene không bị hủy hoại, trên con đường. Con đường càng dài, thì càng có nhiều nỗ lực khai thác mỏ cần phải được thực hiện để đến được lá. Điều này tương tự như các lược đồ hiện có, chẳng hạn như các công thức được sử dụng trong các giao thức bắt nguồn từ Bitcoin.
\newline\indent Kể từ khi một tiêu đề khối bao gồm các khó khăn, tiêu đề một mình là đủ để xác nhận tính toán thực hiện. Bất kỳ khối nào đóng góp vào tổng số tính toán hoặc tổng số khó khăn của một chuỗi.
\newline\indent
Do đó chúng ta xác định tổng số khó khăn của khối B đệ quy như sau:
$$B_t \equiv B'_t+B_d$$
$$B' \equiv P(B_H)$$
Trong đó cho biết khối B, $B_t$ là tổng số khó khăn B' là khối cha và $B_d$ là khó khăn.