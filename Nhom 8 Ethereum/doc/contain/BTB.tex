\documentclass[../cnd.tex]{subfiles}

Các chuỗi khối hợp tiêu chuẩn là một con đường từ gốc đến lá thông qua toàn bộ cây khối. Để có sự thống nhất về con đường đó, chúng ta xác định con đường có tính toán nhiều nhất được thực hiện theo nó, hoặc, con đường có trọng số lớn nhất. Rõ ràng một yếu tố giúp xác định được con đường có trọng số lớn nhất là số khối của lá, tương đương với số khối trên con đường đó, không tính đến khối gene không bị hủy hoại. Con đường càng dài, thì càng cần nhiều nỗ lực để khai thác mỏ và cần phải được thực hiện để đến được lá.
\newline\indent Một khối tiêu đề bao gồm các độ khó,khối tiêu đề đơn lẻ là đủ để xác nhận tính toán cần thực hiện. Bất kỳ khối nào cũng đóng góp vào tổng số tính toán hoặc tổng số độ khó của một chuỗi.
\newline\indent
Do đó chúng ta xác định tổng độ khó của khối B đệ quy như sau:
$$B_t \equiv B'_t+B_d$$
$$B' \equiv P(B_H)$$
Trong đó cho biết khối B, $B_t$ là tổng độ khố B' là khối cha và $B_d$ là độ khó.