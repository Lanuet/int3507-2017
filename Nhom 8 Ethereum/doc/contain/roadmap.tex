\documentclass[../cnd.tex]{subfiles}
Mặc dù việc phát triển phần mềm có thể rất khó dự đoán, các nhà phát triển của Ethereum đã có các dự đoán về các mốc thời gian khá rõ ràng. Cụ thể sẽ có các giai đoạn sau.
	\subsubsection{Frontier Release (2015)}
	Frontier đã hoàn thành một số mục tiêu đúng hạn. Trong giai đoạn này việc giao tiếp chủ yếu được thực hiện trên các dòng lệnh. Các ưu tiên trong giai đoạn này gồm có:
	\begin{itemize}
  		\item Đảm bảo hệ thống mining chạy được
  		\item Hợp pháp hóa đồng ETH như là một loại tiền ảo chính thống
  		\item Phát hành một môi trường để thử nghiệm dapp
  		\item Tạo ra các công cụ để cung cấp ether miễn phí cho các mạng thử nghiệm
  		\item Cho phép các nhà phát triển tải lên và thự thi các hợp đồng thông minh
	\end{itemize}

	\subsubsection{Homestead Release (2016)}
	Giai đoạn này cung cấp thêm một công cụ hữu ích hơn cho việc giao tiếp với Ethereum bằng Mist Browser. Các đặc điểm chính của giai đoạn này bao gồm:
	\begin{itemize}
  		\item Việc đào Ether đã được trả công 100\% giá trị như dự định
  		\item Không có sự ngắt quãng trên toàn mạng
  		\item Gần như bước ra khỏi quá trình thử nghiệm
  		\item Có thêm các tài liệu cho việc sử dụng dòng lệnh và Mist
	\end{itemize}
	
	\subsubsection{Metropolis (2017)}
	Đây là pha thứ hai trong việc phát triển giao thức Ethereum. Bản phát hành này sẽ mở ra việc phát hành chính thức cho Mist với đầy đủ các tính năng. Tiếp theo đó nó sẽ được hỗ trợ bởi các phần mềm bên thứ ba một cách mạnh mẽ. Hơn nữa, Swarm và Whisper sẽ được đưa vào vận hành.

	\subsubsection{Serenity (2018)}
	Đây sẽ là giai đoạn chuyển mô hình từ POW to POS với thuật toán có tên gọi Casper. Việc chuyển dịch này là cần thiết với các ưu diểm được nêu từ trước. Quá trình chuyển dịch này sẽ được thực hiện từ từ chứ không chuyển hẳn sang mô hình mới trong một lần. Việc chuyển dịch này sẽ rất cần sự tính toán cũng như chứng minh tính đúng đắn của nó.