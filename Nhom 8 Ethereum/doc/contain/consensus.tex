\documentclass[../cnd.tex]{subfiles}
Trên một mạng ngang hàng, mọi dữ liệu ở các nút cục bộ đều có thể bị thay đổi túy ý bởi người sở hữu nút đó và khi thực hiện nhân bản thì sẽ có các xung đột và ta khó có thể xác định đâu mới là chuỗi khối đúng. Vì vậy, để đảm bảo sự tin cậy của mạng cũng như chuỗi khối, ta phải có các cơ chế để làm sao tất các nút trên mạng đều đồng thuận khối thêm vào chuỗi đó là khối hợp lệ và sẽ phát hiện ra các hành vi phá hoại mạng bằng cách đưa vào các khối giả. Nói về cơ chế đồng thuận này thì hiện nay phổ biến nhất sẽ có hai cơ chế là chứng nhận công việc (Proof Of Work - POW) và chứng nhận cổ phần (Proof Of Stake - POS). Ngoài ra còn mô hình chứng nhận thẩm quyền (Proof Of Authority - POA) nhưng trước hết ta sẽ đi vào giải thích POW và POS.
	\subsubsection{POW}
	Ta đã quá quen thuộc với việc các thợ mỏ mua các cỗ máy đắt tiền (gồm nhiều card đồ họa) để về đào coin. Việc này chúng tỏ họ đang tham gia vào một mạng lưới tiền ảo sử dụng mô hình POW. POW là một trong các cách để xác định sự đồng thuận của cộng đồng. Ở mô hình hay giải thuật này, để thêm mới một khối và blockchain đồi hỏi phải thực hiện các hàm tính toán rất phức tạp để tạo nên một giá trị băm hợp lệ (khó để tạo ra nhưng rất dễ dàng để xác định nó hợp lệ). Việc giải mã này ngoài việc cần những cấu hình mạnh còn tiêu tốn rất rất nhiều điện năng và gây nguy hại đến môi trường. Cộng đồng sẽ công nhận khối anh tạo ra dựa vào lượng công việc anh đã thực hiện được. Một yếu tố đặc trưng của mô hình này đó là sự xuất hiện của các khối dư thừa (orphan block) do có nhiều người tham gia cùng tạo nên một khối nên sẽ có trường hợp hai người tạo ra hai khối đều hợp lệ nhưng ta chỉ có thể chọn khối đến trước và khối kia (tốn rất nhiều năng lượng tạo ra) bị lãng phí và bỏ đi. Những đặc trưng kể trên phần nào nói lên nhược điểm của giải thuật mà phần lớn các đồng tiền ảo đang sử dụng

Khi độ khó để tạo ra một khối càng ngày càng tăng lên, việc đào ra một khối của các thợ mỏ ngày càng thấp và họ bắt đầu thua lỗ,một số sẽ quyết định từ bỏ hoặc tham gia vào các bể đào (mining pool). Việc này tạo ra các cỗ máy tập trung (trái với các tính chất phân tán mà mô hình mong muốn). Và còn một vấn đề hệ quả nữa là tấn công 51% (51% attack) sẽ được nên ở sau

Nhưng nó phổ biến có nguyên nhân của nó. Giá trị sử dụng của nó là khi một đồng tiền ảo mới được phát hành, lượng tiền chưa được nhiều. Hầu hết các đồng tiền ảo sẽ chọn mô hình này để gia tăng lượng tiền mà vẫn kiểm soát được phần nào lạm phát của nó (điều mà ETH đã làm), tiền chỉ được tạo ra không hề dễ dàng.
	\subsubsection{POS}
	Khi mà POW bộc lộ rõ các yếu điểm thì là lúc các mô hình và các giải thuật khác được đề xuất, trong đó có POS. Thực tế thì POS đã được áp dụng ở một số đồng tiền ảo, tiên phong trong đó là PeerCoin và Ethereum sẽ là cái tên tiếp theo áp dụng mô hình này. Mô hình POS thay vì công nhận công việc của anh bằng sức lực anh bỏ ra (chi cho phần cứng và năng lượng) thì lại công nhận bằng nó bằng cổ phần hay số tiền mà anh đặt cọc vào mỗi khối người đó sinh ra. Và giờ đây, mỗi khối không phải là cuộc chạy đua xem ai giải mã được chính xác và nhanh nhất (có thể gây lãng phí với các khối thừa) mà việc tạo khối được chỉ định cho người nào góp cổ phần nhiều nhất vì hệ thống tự động hiểu rằng nếu anh góp vào nhiều tiền như vậy thì anh cũng sẽ có đủ khả năng tính toán để tạo ra khối mới.Và một điều quan trọng nữa là sẽ không còn tồn tại việc thưởng ETH theo mỗi khối đào được mà chỉ trả cho người tạo khối một khoản tiền vì đã thực hiện giao dịch (transaction fee).

Việc sử dụng POS sẽ dẫn đến các lợi ích như sau. Tiết kiệm năng lượng, giúp mạng lưới phân tán hơn, giúp đảm bảo lợi ích người đào.

	\subsubsection{POA}
	POA khá khác so với hai mô hình đồng thuận trước, trong khi hai mô hình trên vẫn giữ tính ẩn danh của tất cả những người tham gia và sự tin tưởng khó có thể đạt tới sự chắc chắn (tạm tin tưởng vào lượng công việc cũng như cổ phần) thì thay vào đó ta sẽ có các cơ quan được chứng thực một cách hợp pháp và công khai tham gia vào mạng như các thành viên kiểm chứng. Những cơ quan này sẽ sở hữu các tài khoản và các nút được coi là có thẩm quyền, các thành viên trong mạng có thể tin tưởng hoàn toàn họ và họ sẽ đóng vai trò chính trong việc kiểm tra các giao dịch và các khối có hợp lệ hay không. Mạng theo mô hình này sẽ có một số lợi thế vượt trội so với hai mô hình kia như thời gian tạo khối tương đối ngắn, không lãng phí vì không cần phải đào, giúp cải thiện thời gian triển khai, kiểm thử và ngăn chặn các cuộc tấn công spam. Ethereum đã triển khai một mạng thử nghiệm triển khai mô hình này có tên là Kovan, bạn có thể xem chi tiết hơn tại \url{https://github.com/kovan-testnet/proposal}.
	