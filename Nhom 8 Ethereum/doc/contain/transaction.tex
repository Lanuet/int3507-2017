\documentclass[../cnd.tex]{subfiles}
Máy ảo Ethereum (EVM) là một máy chủ toàn cầu mà mọi người có thể sử dụng, với một chi phí nhỏ, có thể trả bằng ether
\subsubsection{Mạng ngân hàng trung tâm}
Chúng ta đang chi rất nhiều tiền cho việc xây dựng, vận hành và bảo trì các hệ thống thanh toán tập trung. Mỗi ngân hàng xây dựng riêng cho mình một hệ thống và việc thanh toán liên ngân hàng sẽ đòi hỏi thêm một khoản phí nữa, chưa kể đến các chi phí để đảm bảo an toàn, an ninh cũng như tính tin cậy cho cả hệ thống. Những chi phí ấy không hề rẻ dẫn đến việc bạn phải chi trả nhiều hơn cho các giao dịch của mình cũng như các giao dịch sẽ bị chậm đi đáng kể khi phải liên kết niều dịch vụ.
\subsubsection{Máy ảo}
Trong ngữ cảnh của Ethereum, đó là một máy tính toàn cầu bao gồm rất nhiều nút cấu thành và chính các nút đó cũng là các máy tính. Nói chung, máy ảo là mô phỏng một hệ thống máy tính này bằng một hệ thống máy tính khác. Việc mô phỏng này trên EVM sử dụng cả phần cứng và phần mềm, mỗi nút trong mạng có thể chạy bắt kì hệ điều hành nào.
\subsubsection{Vai trò của EVM}
EVM tạo ra môi trường để chạy các chương trình bất kì (được gọi là các hợp đồng thông minh) được viết bằng ngôn ngữ Solidity. Những chương trình này là xác định đầy đủ và đảm bảo là được thực hiện nếu bạn trả đủ chi phí co giao dịch.
Các chương trình viết bằng Solidity có khả năng thực hiện tất cả các nhiệm vụ có thể thực hiện được bằng máy tính. Khi một hợp đồng được triển khai bằng việc tải lên từ một nút của mạng, các bản sao của hợp đồng sẽ được phát tán ra các nút khác. Khi có yêu cầu chạy hợp đồng, các nút trong mạng sẽ chạy một các độc lớp với cùng một mã của hợp đồng. Tuy nó cho thấy sự song song hóa rất cao, nhưng nó cũng mang lại sự dư thừa không hề nhỏ.
